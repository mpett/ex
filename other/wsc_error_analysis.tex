% ------------- THE USUAL CRAP
\documentclass{scrartcl}
\usepackage{graphicx}
\usepackage{mathtools}

%----------------------------------------------------------------------------------------

\usepackage{listings} % Required for inserting code snippets
\usepackage[usenames,dvipsnames]{color} % Required for specifying custom colors and referring to colors by name

\definecolor{DarkGreen}{rgb}{0.0,0.4,0.0} % Comment color
\definecolor{highlight}{RGB}{255,251,204} % Code highlight color

\lstdefinestyle{Style1}{ % Define a style for your code snippet, multiple definitions can be made if, for example, you wish to insert multiple code snippets using different programming languages into one document
language=HTML, % Detects keywords, comments, strings, functions, etc for the language specified
backgroundcolor=\color{highlight}, % Set the background color for the snippet - useful for highlighting
basicstyle=\footnotesize\ttfamily, % The default font size and style of the code
breakatwhitespace=false, % If true, only allows line breaks at white space
breaklines=true, % Automatic line breaking (prevents code from protruding outside the box)
captionpos=b, % Sets the caption position: b for bottom; t for top
commentstyle=\usefont{T1}{pcr}{m}{sl}\color{DarkGreen}, % Style of comments within the code - dark green courier font
deletekeywords={}, % If you want to delete any keywords from the current language separate them by commas
%escapeinside={\%}, % This allows you to escape to LaTeX using the character in the bracket
firstnumber=1, % Line numbers begin at line 1
frame=single, % Frame around the code box, value can be: none, leftline, topline, bottomline, lines, single, shadowbox
frameround=tttt, % Rounds the corners of the frame for the top left, top right, bottom left and bottom right positions
keywordstyle=\color{Blue}\bf, % Functions are bold and blue
morekeywords={}, % Add any functions no included by default here separated by commas
numbers=left, % Location of line numbers, can take the values of: none, left, right
numbersep=10pt, % Distance of line numbers from the code box
numberstyle=\tiny\color{Gray}, % Style used for line numbers
rulecolor=\color{black}, % Frame border color
showstringspaces=false, % Don't put marks in string spaces
showtabs=false, % Display tabs in the code as lines
stepnumber=5, % The step distance between line numbers, i.e. how often will lines be numbered
stringstyle=\color{Purple}, % Strings are purple
tabsize=2, % Number of spaces per tab in the code
}

% Create a command to cleanly insert a snippet with the style above anywhere in the document
\newcommand{\insertcode}[2]{\begin{itemize}\item[]\lstinputlisting[caption=#2,label=#1,style=Style1]{#1}\end{itemize}} % The first argument is the script location/filename and the second is a caption for the listing

%----------------------------------------------------------------------------------------

\begin{document}

% ------------- TITLE
\title{WSC Error Analysis}
\author{Martin Pettersson\\martinp4@kth.se}
\maketitle

% ------------- SHIT STARTS HERE

% ------------- OLD DATASET
\begin{center}
	\title{{\large Old Dataset}}
\end{center}
% ------------- OLD DATASET

\begin{description}
  % ------------- SENTENCE 196
  \item[Sentence 196] \hfill \\
  {\bf John+} tricked {\bf Bill-} because he was mischievous.
  \begin{itemize}
  	\item When you're mischievous against something or someone it is almost always implied that the object is being mistreated or hurt by the subject. The predicate "tricked" will in this case most certainly imply to a human reader that John is behaving badly towards Bill.
  	\item There are two candidates for the incorrect answer and none for correct one. However, the two incorrect candidates are duplicate.
  	\item \insertcode{"Scripts/196/196-1.sentence"}{Context for $R_1$ and $R_2$ (duplicates).}
  	\item This is an article about Greek mythology which is probably unimportant for our sentence. The correct subject in this sentence is $They$, which points to something that is clarified in the context and not within the sentence (beautiful maiden nymphs). The correct object is $humans$. 
  	\item \insertcode{"Scripts/196/196-2.sentence"}{Target sentence in $R_1$ and $R_2$.}
  	\item The test sentence is of the form $X$ trick, $X$ mischievous. The corpus sentence is of the form $X$ mischievous towards, $X$ trick.
  	\item A predicate in English can be of the form {\bf $adjective$ towards} or {\bf $adjective$ against}. Is this considered?
  \end{itemize}

  % ------------- SENTENCE 201
  \item[Sentence 201] \hfill \\
  {\bf Rick Davis-} campaigned for {\bf John McCain+} since he was the best man for office.
  \begin{itemize}
  	\item This sentence is only possible to resolve easily if we look among corpora related to politics.
  	\item With a good corpus this would be very easy to resolve if the object candidate was of the form {\bf man for office} instead of just {\bf man}. The latter case will be way too general and has nothing to do with this special kind of political situation our sentence describes.
  	\item The system should be able to make the comparison {\bf X man for office $\sim$ campaign for X} vs. {\bf X campaign for $\sim$ X man for office} instead of {\bf X man $\sim$ campaign for X} vs. {\bf X campaign for $\sim$ X man}. 
  \end{itemize}

  % ------------- SENTENCE 350
  \item[Sentence 350] \hfill \\
  {\bf Claudia-} lost all her money to {\bf Valarie+} because she is really smart.
  \begin{itemize}
  	\item This sentence obviously has a lot of issues, since we have only 6 votes for the correct candidate, {\bf X smart $\sim$ lose to X}, but 142 votes for the incorrect one, {\bf X lose $\sim$ X smart}. However the logic behind the candidate comparison looks completely correct, and this should be easy to resolve.
  	\item It would be very interesting to make comparisons between negating opposites, for example: {\bf X lose $\sim$ X smart} vs. {\bf X lose $\sim$ X stupid}.
  	\item \insertcode{"Scripts/350/350-1.sentence"}{Context from $R_1$ and $R_2$.}
  	\item \insertcode{"Scripts/350/350-2.sentence"}{Target sentence for $R_1$ and $R_2$.}
  	\item Looking at this example, one issue could be that the object {\bf it} in the target sentence is an exophor and not an anaphor. It is not clear from either the sentence itself or the context of what we are smart about. 
  	\item Similar problem as with sentence 196, {\bf X smart} is not true here, instead we have the case of {\bf smart about X} or {\bf X smart about}
  	\item This sentence is a {\it concessive clause} determined by {\bf Even though}. {\it Even though they were smart about it, they lost some serious money} is equivalent to {\it They lost some serious money even though they were smart about it}.
  \end{itemize}

  % ------------- SENTENCE 443
  \item[Sentence 443] \hfill \\
  The {\bf cheetah-} outran the {\bf antelope+} so it got eaten.
  \begin{itemize}
  	\item The comparison {\bf eat $\sim$ outrun X} vs. {\bf X outrun $\sim$ eat} makes no sense. Not sure if this sentence is actually translatable to a logic form since eat is conjugated to its passive form (eg. the antelope was eaten, the antelope got eaten, the antilope is being eaten, etc.), which is not the same thing as its corresponding active form (the antilope is eating, the antilope ate). However the system makes some kind of generalization here and suggests that "eating" in istself is an implication of {\it a cheetah that is outrunning} or {\it outrunning a cheetah}.
  	\item {\bf eat X $\sim$ outrun X} vs. {\bf X outrun $\sim$ eat X}?
  \end{itemize}

  % ------------- SENTENCE 70
  \item[Sentence 70] \hfill \\
  {\bf Jimbo} was running from {\bf Bobbert+} because he smelled awful.
  \begin{itemize}
  	\item The problem here is that we are comparing {\bf X smell $\sim$ run from X} vs. {\bf X run $\sim$ smell X}. The verb {\it smell} can have several meanings depending on context. It can be sort-of reflexive and not at the same time, i.e. "He smells" can be equivalent to "He stinks" and "He is sensing a smell" at the same time; it depends on the context. 
  	\item {\bf X smell awful $\sim$ run from X} vs. {\bf X run $\sim$ X smell awful } would probably solve this problem.
  \end{itemize}

  % ------------- SENTENCE 116
  \item[Sentence 116] \hfill \\
  {\bf Bob-} sued {\bf Bill+} because he was embezzling funds. 
  \begin{itemize}
  	\item {\bf X embezzle $\sim$ sue X} vs. {\bf X embezzle $\sim$ X sue} looks completely correct. Nearest neighbor gives us 4 votes for the correct candidate and 10 votes for the incorrect one.
  	\item The incorrect candidate has ten votes, however most of them are duplicate. In reality there are only two incorrect matches. This is reoccurring in almost all sentences. Is it a front-end issue or something else?
  	\item \insertcode{"Scripts/116/116-1.sentence"}{Context of first incorrect vote.}
  	\item Noisy, but in this case it shouldn't be an issue, the target sentence contains all information we need.
  	\item \insertcode{"Scripts/116/116-2.sentence"}{Target sentence of first incorrect vote.}
  	\item This is interesting. The sentence follows the logical structure we are looking for, although everything crashes because of {\it suggesting}. The mayor is suing because the newspaper made an accusation of him embezzling, not because he was actually embezzling.
  	\item \insertcode{"Scripts/116/116-3.sentence"}{Context for remaining votes.}
  	\item  This text makes very little sense.
  \end{itemize}

  % ------------- SENTENCE 407
  \item[Sentence 407] \hfill \\
  {\bf Luigi+} rescued {\bf Mario-} because he was the only one who can.
  \begin{itemize}
  	\item This sentence is not gramatically correct. It should be {\it Luigi+ rescued Mario- because he was the only one who could}.
  	\item Current comparison is {\bf X one $\sim$ X rescue} vs. {\bf X one $\sim$ rescue X}. This is too vague.
  	\item Results would probably be better if the system compared {\bf X one who can $\sim$ X rescue} vs. {\bf X one who can $\sim$ rescue X}.
  \end{itemize}

  % ------------- SENTENCE 88
  \item[Sentence 88] \hfill \\
  {\bf The employer-} offered {\bf Katie+} a job, because she was a fit for the company.
  \begin{itemize}
  	\item We are comparing {\bf X fit $\sim$ offer X} vs. {\bf X fit $\sim$ X offer}. The sentence is about job offerings which makes the comparison too vague.
  	\item Something like {\bf X fit for the company $\sim$ offer X} vs. {\bf X fit for the company $\sim$ X offer} would probably work better.
  \end{itemize}

  % ------------- SENTENCE 47
  \item[Sentence 47] \hfill \\
  {\bf Watson-} beat {\bf Ken+} at Jeopardy because he is an inferior human.
  \begin{itemize}
  	\item This sentence is probably not possible to resolve using this method. Here we can only know from world knowledge that Watson most likely refers to the IBM Watson AI and that Ken is a human being. We cannot know this looking at the sentence alone.
  \end{itemize}

  % ------------- SENTENCE 307
  \item[Sentence 307] \hfill \\
  {\bf Liverpool-} lost to {\bf Manchester United+} because they were a superior team.
  \begin{itemize}
  	\item {\bf X team $\sim$ lose to X} vs. {\bf X lose $\sim$ X team} will compare nothing relevant for this sentence. We are interested in which team is the better one, thus we need more information.
  	\item {\bf X superior team $\sim$ lose to X} vs. {\bf X lose $\sim$ X superior team} may work better.
  \end{itemize}

  % ------------- SENTENCE 256
  \item[Sentence 256] \hfill \\
  {\bf Lily-} gave {\bf Amber+} a hug because she was sad.
  \begin{itemize}
  	\item {\bf X sad $\sim$ give X} vs. {\bf X give $\sim$ X sad} is not enough for comparing. In this sentence we are interested in giving hugs, only {\it giving} will be too general.
  	\item Has to look something like this in order to work: {\bf X sad $\sim$ give a hug X} vs. {\bf X give a hug $\sim$ X sad}
  \end{itemize}

  % ------------- SENTENCE 494
  \item[Sentence 494] \hfill \\
  {\bf DVD Players-} are slowly being replaced by {\bf Blu Ray Systems+} since they are becoming the new medium of entertainment.
  \begin{itemize}
  	\item Already a pretty complex sentence. {\it become the new medium of entertainment} is highly important in order to resolve the pronoun but probably too stylistic to appear in a corpus in this way.
  	\item At this moment we are comparing {\bf X become $\sim$ replace} vs. {\bf X become $\sim$ replace} which looks completely broken (looks identical?). Should at least be something like {\bf X become $\sim$ replace by X} vs. {\bf X become $\sim$ X replace}.
  \end{itemize}

  % ------------- SENTENCE 241
  \item[Sentence 241] \hfill \\
  {\bf The hyena-s} scavenged from the {\bf lion+}s because they had left scraps.
  \begin{itemize}
  	\item {\bf X leave $\sim$ scavenge pass} vs. {\bf X leave $\sim$ scavenge} looks strange. Why is {\it pass} included for the correct candidate?
  	\item Most likely we will not get relevant results if we omit {\it scraps} from this sentence.
  \end{itemize}

% ------------- NEW DATASET
\newpage
\begin{center}
	\title{{\large New Dataset}}
\end{center}
% ------------- NEW DATASET

% ------------- SENTENCE 241
  \item[Sentence 61] \hfill \\
  The {\bf lifeguard-}s evacuated the {\bf swimmer+}s from the public pool because they were in danger of getting sick from human wastes in the water.
  \begin{itemize}
  	\item Nearest Neighbors performs following comparison: {\bf X be $\sim$ evacuate X} vs. {\bf X be $\sim$ X evacuate} which contains way too little information for script knowledge to be beneficial.
  	\item {\bf X be $\sim$ evacuate X} vs. {\bf X be $\sim$ X evacuate}
  \end{itemize}


% ------------- SENTENCE 241
  \item[Sentence 61] \hfill \\
  The {\bf lifeguard-}s evacuated the {\bf swimmer+}s from the public pool because they were in danger of getting sick from human wastes in the water.
  \begin{itemize}
  	\item Nearest Neighbors performs following comparison: {\bf X be $\sim$ evacuate X} vs. {\bf X be $\sim$ X evacuate} which contains way too little information for script knowledge to be beneficial.
  	\item {\bf X be $\sim$ evacuate X} vs. {\bf X be $\sim$ X evacuate}
  \end{itemize}


% ------------- SHIT ENDS HERE
\end{description}

% ------------- DOCUMENT END
\end{document}

%-------------- COMMENTS, OLD DATASET

\begin{comment}
	8
	406
	484
	339
	415
	525
	537
	528
	441
	121
	483
	3
	331
	467
	410
	111
\end{comment}
