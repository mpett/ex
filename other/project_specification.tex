% ------------- THE USUAL CRAP
\documentclass{article}
\usepackage{graphicx}

\begin{document}

\title{Project Proposal for Master's Thesis}
\author{Martin Pettersson (martinp4@kth.se)}
\maketitle

% ------------- TITLE
\section{Preliminary Thesis Title}
"Resolving Difficult Pronouns Using Script Knowledge"\\
"Attempting The Winograd Schema Challenge Using Script Knowledge"

% ------------- BACKGROUND
\section{Background and Motivation}
A proposed alternative to the famous Turing Test are resolution of so-called Winograd Schemas. The reason behind this is to pursue a more reliable way to test human thinking behaviour in machines without using deception techniques, naive scripts or by cheating. The basic idea is to present a machine with questions that are easy to answer for English-speaking adults, but still so ambigous that a large corpus probably would not help much. 

An example of a Winograd Schema is presented as follows:
\\ \\
{\it The trophy would not fit in the brown suitcase because it was too big (small). What was too big (small)?}
\begin{itemize}
  \item The trophy
  \item The suitcase
\end{itemize}
Being part of a bilateral exchange program in Japan, I currently work as a research student in a laboratory at Tohoku University that is specialized in NLP. One of the laboratory's research groups, focused on Knowledge Intensive Artificial Intelligence, has developed a system able to conduct large-scale acquisition of narrative event pairs from a huge corpus. At this moment they are using a case-based kNN framework in order to discover potential impacts using script knowledge in coreference resolution. In the beginning of October this year they will participate in The Winograd Schema Challenge, an AI competition hosted by the non-profit organization Commonsense Reasoning.

With this thesis I aim to further explore the benefits of using script knowledge in coreference resolution by analyzing, evaluating and finally attempt to make improvements to their current system.
\subsection{Location}
The project will be carried out at Inui-Okazaki Lab of Communication Science, Tohoku University Graduate School of Information Sciences, Sendai-shi, Miyagi, Japan.
% ------------- SCIENTIFIC QUESTION
\section{Scientific Question}
The thesis will attempt to answer the question if script knowledge could be useful to improve existing models in coreference resolution.
\subsection{Current Research}

\subsection{Methodology}
I have already spent a proper amount of time researching the subject by reading literature and looking up previous studies to a point where I feel confident.

For start I will perform error analysis on the current system. This will be done by looking at the training data, their input dataset and the current results. While doing this I will continously suggest improvements and new features in collaboration with a research group working on different areas of the same project. When this is done a list with a number of experiments and implementations will be written to best fit the remaining project time.
\subsection{Hypothesis}
Script knowledge may be proven to be useful in certain cirumstances as a method for improving already existing models in coreference resolution.
\subsection{Evaluation}
The thesis project may be considered complete when:
\begin{itemize}
	\item All suggested improvements to the current system have been implemented and their respective results have been recorded.
	\item A failure. This would basically mean that all possible methods of performing suggested experiments have been attempted under guidance and surveillance of the mentors at the lab.
\end{itemize}
% ------------- PERSONAL BACKGROUND
\section{Personal Background}
To begin with I have had a general interest in AI and language since high school. I enrolled the international profile of the computer science program at KTH in 2010. In 2013 I began studying the master's program in computer science with the Language Engineering track as well as continuing studies in Japanese. Courses I have taken at KTH that are relevant to this project include Artificial Intelligence, Algorithms, Data Structures and Complexity, Advanced Algorithms, Markov Models and Mathematical Statistics among others. 

I began my exchange studies at Tohoku University in September 2014. During the first five months I have spent my time mainly learning the fundamentals of Natural Language Processing and Machine Learning with guidance from my supervisors at the lab. I have also made some attempts to improve my Japanese skills further.
% ------------- SUPERVISOR AT TOHOKU UNIVERSITY
\section{Supervisor at Tohoku University}
Professor Kentaro Inui is the official supervisor and will examine the final project. I have been placed in a research group with people of varying background that I may turn to for daily practical issues and guidance. The purpose of the research group is to provide a continous feedback loop.

% ------------- AVAILABLE RESOURCES
\section{Available Resources}
The laboratory at Tohokudai has access to several research paper databases that are relevant to Natural Language Processing and Machine Learning. I have also been given access to high-performance computing nodes at their department where experiments may be performed.


\end{document}
